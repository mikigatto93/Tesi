
%**************************************************************
% Acronimi
%**************************************************************
%\newacronym[description={\glslink{apig}{Application Program Interface}}]
%    {api}{API}{Application Program Interface}
%
%\newacronym[description={\glslink{umlg}{Unified Modeling Language}}]
%    {uml}{UML}{Unified Modeling Language}

%\newacronym[description={Key Performance Indicator}]
%    {kpi}{KPI}{Key Performance Indicator}


%**************************************************************
% Glossario
%**************************************************************

%\newglossaryentry{KPI}
%{
%    name={KPI},
%    text=KPI,
%    sort=kpi,
%    description={I KPI, acronimo di \emph{Key Performance Indicators} sono un insieme di misure quantificabili che un’azienda utilizza per valutare le sue prestazioni nel tempo}
%}


\newglossaryentry{API}
{
    name={API},
    text=API,
    sort=api,
    description={In informatica con il termine \emph{Application Programming Interface API} (ing. interfaccia di programmazione di un'applicazione) si indica ogni insieme di procedure disponibili al programmatore, di solito raggruppate a formare un set di strumenti specifici per l'espletamento di un determinato compito all'interno di un certo programma. La finalità è ottenere un'astrazione, di solito tra l'hardware e il programmatore o tra software a basso e quello ad alto livello semplificando così il lavoro di programmazione}
}

\newglossaryentry{UML}
{
    name={UML},
    text=UML,
    sort=uml,
    description={In ingegneria del software \emph{UML, Unified Modeling Language} (ing. linguaggio di modellazione unificato) è un linguaggio di modellazione e specifica basato sul paradigma object-oriented. L'\emph{UML} svolge un'importantissima funzione di ``lingua franca'' nella comunità della progettazione e programmazione a oggetti. Gran parte della letteratura di settore usa tale linguaggio per descrivere soluzioni analitiche e progettuali in modo sintetico e comprensibile a un vasto pubblico}
}

\newglossaryentry{PoC}
{
    name={PoC},
    text=PoC,
    sort=PoC,
    description={Un PoC, acronimo di \emph{Proof of Concept} (“prova di concetto”) non è altro che un test o prova (ad esempio in ambiente informatico può essere un piccolo programma) che ha lo scopo di determinare la fattibilità di un'idea o di verificare che l'idea funzionerà come previsto.}
}

\newglossaryentry{ERP}
{
    name={ERP},
    text=ERP,
    sort=ERP,
    description={ERP è l'acronimo di Enterprise Resource Planning e ci si riferisce a un tipo di software che le organizzazioni utilizzano per gestire le attività quotidiane di business, come ad esempio contabilità, project management, gestione del rischio e operazioni relative all'approvvigionamento. 
I sistemi ERP mettono in relazione tra loro un insieme di processi di business e ne consentono lo scambio di dati. Grazie alla raccolta di dati transazionali condivisi provenienti da diverse fonti dell'organizzazione, i sistemi ERP eliminano la duplicazione dei dati e ne garantiscono l'integrità tramite un'unica fonte di informazioni.}
}
