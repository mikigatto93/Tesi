% !TEX encoding = UTF-8
% !TEX TS-program = pdflatex
% !TEX root = ../tesi.tex

%**************************************************************
\chapter{Descrizione dello stage}
\label{cap2}
\section{Scopo dello stage}
L'idea dello stage nasce dalla necessità di sviluppare una nuova interfaccia grafica front-end per un sistema di raccomandazione per le istanze di processi aziendali.
\\
Il back-end del sistema è stato precedentemente sviluppato dal dottorando Alessandro Padella, come parte del suo lavoro di tesi di dottorato (\cite{paper-padella}). 
\\
Il back-end è un PAR (vedasi \autoref{subsec:par-ref}) che implementa l'analisi prescrittiva, e ha lo scopo di raccomandare le migliori opzioni di decisione nell'ambito dei processi aziendali. Per fare questo vengono sfruttati, in particolare, i risultati dell'analisi predittiva, algoritmi di ottimizzazione e intelligenza artificiale. Le opzioni vengono individuate in termini di KPI (tempo totale di esecuzione e numero di volte in cui una specifica attività si verifica). Infine il sistema genera anche le spiegazioni (sotto forma di Shapley Values) per le raccomandazioni individuate.
\\
Lo scopo sarà quindi di sviluppare un'interfaccia grafica per il sistema, che si occupi di tutte le fasi necessarie: 
\begin{itemize}
\item caricamento dell'event log storico su cui allenare il modello;
\item selezione delle opzioni relative all'allenamento del modello desiderate dell'utente;
\item caricamento dell'event log incompleto su cui fare le predizioni;
\item visualizzazione delle predizioni, delle raccomandazioni e delle relative spiegazioni.
\end{itemize}

\section{Contenuti formativi}
Durante il progetto di stage sono state approfondite le conoscenze nei seguenti ambiti:
\begin{itemize}
\item Process mining; 
\item Progettazione e sviluppo di interfacce grafiche;
\item Framework per la data visualization;
\end{itemize}

\section{Obiettivi}
\label{sec:stage-obj}
Si farà riferimento agli obiettivi secondo le seguenti notazioni:
\begin{itemize}
	\item O per i requisiti obbligatori, vincolanti in quanto obiettivo primario richiesto dal committente;
	\item D per i requisiti desiderabili, non vincolanti o strettamente necessari,
		  ma dal riconoscibile valore aggiunto;
	\item F per i requisiti facoltativi, rappresentanti valore aggiunto non strettamente 
		  necessario.
\end{itemize}
Le sigle precedentemente indicate saranno seguite da un numero sequenziale, identificativo dell'obiettivo.
\\
Lo stage prevedeva inizialmente i seguenti obiettivi:
\begin{itemize}
	\item Obbligatori
	\begin{itemize}
		\item \underline{\textit{O01}}: Produzione di un mock-up che illustra le diverse schermate che costituiscono l'interfaccia grafica;
	 \item \underline{\textit{O02}}: Produzione di diagrammi per i casi di uso considerati;
	\item \underline{\textit{O03}}: Documento di almeno quattro pagine che riassume i diversi framework analizzati con relativi pro e contro di ognuno e discute il framework scelto, motivando la scelta;
	 \item \underline{\textit{O04}}: Definizione e utilizzo delle \gls{API} di back-end, con il supporto di diagrammi di classe e diagrammi di sequenza;
\item \underline{\textit{O05}}: Sviluppo della prima versione del codice per lo sviluppo dell'interfaccia grafica e per la comunicazione con le \gls{API} di back-end;
\item \underline{\textit{O06}}: Definizione ed esecuzione dei casi di test (di unità, di interfaccia, ecc.);
\item \underline{\textit{O07}}: Sviluppo della versione finale del codice per lo sviluppo dell'interfaccia grafica e per la comunicazione con le \gls{API} di back-end;
\item \underline{\textit{O08}}: Rilascio e deployment del codice finale dopo ultima validazione.
	\end{itemize}
	
	\item Desiderabili 
	\begin{itemize}
		\item \underline{\textit{D01}}: Validazione delle \gls{API} di back-end. 
	\end{itemize}
	
	\item Facoltativi
	\begin{itemize}
		\item \underline{\textit{F01}}: Stesura documentazione.
	\end{itemize} 
\end{itemize}


\clearpage

\section{Pianificazione del lavoro}
Il periodo di svolgimento dello stage è stato dal 10 ottobre 2022 al 10 febbraio 2023, svolto in part-time, per la durata totale di 320 ore. Il part-time consisteva nello svolgere 4 ore di lavoro al giorno, per 5 giorni alla settimana, per un totale di 20 ore di lavoro settimanali.
Infine sono state considerate le 2 settimane di ferie natalizie.
\\
La pianificazione delle attività settimanali era la seguente:
  \begin{itemize}
	
        \item[] \textbf{Dalla prima alla quarta settimana (80 ore)} 
	\begin{itemize}
            	\item{Studio del problema};
	 	\item{Sviluppo di mock-up e casi d'uso};
	\end{itemize}
       

        \item[] \textbf{Dalla quinta alla settima settimana (60 ore)} 
	\begin{itemize}
            	\item{Ricerca e studio di framework grafici e visualizzazioni dati};
	 	
	\end{itemize}
        
	\item[] \textbf{Dall'ottava alla nona settimana (40 ore)}
	\begin{itemize}
            	\item {Studio e definizione di \gls{API} per il back-end};
	\end{itemize}
        
       \item[] \textbf{Dalla decima alla dodicesima settimana (60 ore)} 
	\begin{itemize}
            	\item {Sviluppo di un primo prototipo};
	\end{itemize}
        
        \item[] \textbf{Tredicesima Settimana (20 ore)} 
	\begin{itemize}
            	\item{Validazione del prototipo};
	 	
	\end{itemize}
        
        
        \item[] \textbf{Dalla quattordicesima alla quindicesima settimana (40 ore)} 
        
	\begin{itemize}
            	\item{Sviluppo del prodotto finale (che elimina le criticità osservate nei test effettuati nella tredicesima settimana)};

	\end{itemize}

         \item[] \textbf{Sedicesima Settimana (20 ore)}   
	\begin{itemize}
            	\item{Validazione e collaudo finale};
	\end{itemize} 

 \end{itemize}


\section{Variazioni alla pianificazione originale}
Durante lo svolgimento dello stage sono state applicate delle modifiche alla pianificazione e agli obiettivi da raggiungere. Tutto ciò in relazione al framework che è stato scelto per svolgere il progetto, che sarà approfondito nella \autoref{sec:analisi-framework}. Grazie ad esso l'\gls{API} per la comunicazione tra front-end e back-end si è resa non necessaria.
In particolare l'attività pianificata nell'ottava e nella nona settimana è stata sostituita con la seguente attività: sviluppo di un primo prototipo;
\\
Per quanto riguarda gli obiettivi sono stati modificati nel seguente modo:
	\begin{itemize}
    \item \underline{\textit{O04}} è stato sostituito con: Definizione architettura generale con il supporto di diagrammi di classe;
    \item \underline{\textit{O05}} e \underline{\textit{O07}} sono stati aggiornati, rimuovendo la parte relativa allo sviluppo delle \gls{API};
    \item \underline{\textit{D01}} è stato sostituito con: Implementazione funzionalità multiutente.
	\end{itemize}


\section{Organizzazione dello stage}
Una volta ogni due settimane sono stati organizzati incontri diretti con il tutor esterno, il Prof. Massimiliano de Leoni, per verificare lo stato di avanzamento del progetto, chiarire eventualmente gli obiettivi o i dubbi sorti, affinare la ricerca e aggiornare il piano di lavoro. Il tutor esterno è stato coadiuvato dal Dott. Alessandro Padella, che si è reso disponibile con frequenza anche settimanale per dare supporto nello sviluppo del progetto.



 

